\chapter{Implementation}\label{C:impl}

\section{UI Implementation}\label{imp_ui} 

I have chosen to use javascript with the d3 visualisation library \cite{bostock2011d3} for the client portion of the system. This was chosen as d3 is a very well supported visualisation library offering excellent support for dynamic data, animations, graph layouts and highly customisable charting. Javascript, in addition to being required for D3, is the de-facto standard for interactive web applications. 
In addition JQuery and JQuery ui were used, as these frameworks allow for consistent behaviour across many different browsers, by abstracting away the cross browser support code into the framework. Further, JQuery offers significant support for DOM manipulation in simple and powerful ways. JQueryUI provides implementations of many common desktop UI widgets in forms suitable for web browsers (Javascript, HTMLand css). These widgets greatly simplified UI development, as they can simply be dropped in place and used. JQueryUI also provides CSS files and tools allowing easy tweaking or theming of the tool.
While this required learning a new language, javascript is a relatively simple language, and this did not pose a significant difficulty. Learning javascript was made significantly easier by extensive documentation and familiar syntax and somewhat familiar semantics. 

Filtering functions are currently somewhat limited, with options to show only data for a given username, IP addres, and toggle the display of each class of event. These filters can be freely combined. IP and username filters support wildcards, \_ matches a single character, \% matches 1 or more characters. 

This is a purely visual change, with no change in data, this can be seen as mousing over the bins shows the same summary information. These filters were implemented without altering the selected data to support exploration of potential attacks. ie: when looking for a brute force attack against root, I would hide failed attacks, as there are vastly fewer successful root logins, Then use the mouseover breakdown to check which (if any) bins had a suspiciously high failure rate. This would not be possible were the data to be filtered on the server side, as currently implemented the statistical summary would lose details.

Both date boxes are linked to datepicker tools, which allow for chosing a date from a calendar, and setting time through sliders. These values may be freely edited without reloading data. The Jump to dates button will trigger a reload of data. 
Binlength (Immediately below date controls) allows chosing the length of each timebin. the dropdown controls units (from seconds to years), while the textbox accepts a number of units.

Particular issues with use of javascript included issues with timezone handling, and dynamic typing in combination with implicit variable declaration. In javascript, there are two usable timezones, system local time, and GMT. Other timezones are not exposed. This caused severe issues with the datetime picker controls as there is no way to ensure that the local computer time matches server time. There was no resolution to this issue found, other than to give up and let datetimes rendered by the browser exist in host time, rather than matching the timezone logs are recorded in. A significant and unresolvable issue is caused by this behaviour. When the logging server's timezone does not match that of the tool user's machine the dates and times displayed on the timeline scales will not match those shown in the raw logs. This is a significant source of confusion for users.


Dynamic typing and implicit variable declaration, while sometimes useful, allow for far easier propagation of invalid values, often resulting in abnormal behaviour in areas widely seperated from the source. This proved to be a significant complication to debugging, for limited gain in my opinion. 

As cross browser implementation of the HTML5 history api has significant implementation specific quirks, I have chosen to use a javascript library to work around these issues. History.js. This library has some peculiarities in its handling of history which diverge from the HTML5 spec, cross browser consistency overwhelms these issues however.  The largest issues with history.js is that this library forces a popstate event for every state change. The popstate event is used to notify the page of changes to the history which should be handled. The HTML5 spec calls for this to occur only when history navigation actions are triggered by the user.

This behaviour causes an unneccessary data reload when visibility of event types are changed. however, as the data required for even the largest datasets tested is measured in the 10's of kilobytes, this is not a significant issue. 

\section{Database}\label{imp_db}
Few issues were encountered with the MySql RDBMS used for the project. Two major issues arose, both causing severe performance problems for the parsing tool. One issue was related to index use for range queries, the other represented a limitation of the default java database connection system. 

GeoIp data presented a significant problem during the implementation phase of this project. I made use of the free GeoLite geoIp database, which is provided as a pair of CSV files. IP addresses in this database are represented as 32 bit unsigned integers, with a start and end column defining a range.

The extreme slowness of these queries caused unacceptable performance for the parser due to use of GeoIP information during location clustering. For batch loads potentially tens of thousands of queries would be made to the GeoIP table. Honeynet data consists of 35006 rows, of which 20469 rows resulted in GeoIp lookup. In the ECS dataset, 14885 of 75K rows resulted in GeoIp lookups. 

This design has some unfortunate implications for query efficiency, as there are approximately 2 million ranges (Exact number depending on dataset and version). Range queries across two columns (constant between ColumnA and ColumnB) cannot use an index in most RDBMS systems. This results in very poor performance for queries.
Some simple optimizations are possible, assuming that ranges are non overlapping, and exhaustive. (ie: every address is in exactly 1 range)  With this assumption, a simple query selecting the first range where IP greater than or equal start, in increasing order by start will always select the matching range.

However it's uncertain if the ranges in the dataset are both exhaustive and non overlapping. Checking this property would take significant time and effort, and would need to be repeated  every time the dataset is updated (Monthly). Due to these issues, an alternative approach was sought, and found in the spatial extensions.

Using spatial extensions allows encoding the start and end numbers for IP ranges as a single shape, either line segment or square. Once so encoded, an IP can be range checked by creating a point from the address, and checking if the point is in any shapes. While this shape checking test is slower than a simple inequality test, the ability to use indexes improves performance significantly. Without spatial extensions, a range query may take many seconds, With spatial encoding this dropped to under 0.1 seconds.  

One drawback to spatial extensions for MySql is that they are not supported in the innoDB storage engine, which is the only MySql storage engine to offer transaction support. This is not a significant issue for GeoIP data, where all accesses are reads, exept for a complete table reload each time that the dataset is updated. This reload can be performed in under 5 minutes using batch loading tools available with MySql. As GeoIp databases are often updated only monthly or less frequently, this is an acceptable limitation. 

Shifting to spatially encoded ranges for GeoIp lookups was responsible for a significant improvement in loading times for batch loads. A further significant improvement was achieved by collecting groups of 1000 rows per insert transaction, as transaction overhead for the default JDBC autocommit behaviour was negatively impacting performance. 
 
\section{Parser}

The log parsing tool was implemented in two layers. A log reader, and database writer/analyser. 
The reader layer is responsible for reading any number of log files into a sorted list of events.
This list of events is then passed to the analyser/writer layer, which is responsible for checking connection attempts to see if the location and time are frequently used by the user and writing the analysed metadata to the datastore.

The writer layer is tightly coupled to both the database schema, and RDBM systems specifically as it makes extensive use of the JDBC api for communicating with RDBMS's.
Internal representation of log entries is somewhat coupled to the JDBC API, as each entry is responsible for binding its data to the prepared statements used by the writer.
This could be worked around with minimal changes so long as the new datasource implemented the same API.

Few significant issues were encountered in writing the parser. 
The majority of trouble encountered with the parser was in analysing connection attempts. checking weather a location  is one that is well known is extremely simple, though had significant performance issues discussed in \ref{imp_db}
Time clustering has proven to have ongoing issues, where overlaps in stored time ranges are not properly detected. This error arises where the number of successful login attempts is not correctly aggregated across all time sections which overlap the login attempt. This causes a significant number of false positives for unusual times. I believe that this bug is caused by an error in my SQL queries, though at time of writing this remains unresolved. 

\section{Server}

The server code was implemented as designed. The well documented servlet API, and clear tomcat documentation were very helpful here. Tomcat proved to be an excellent tool for the task.

Due to addition of new filtering options later in the implementation of the servers, it was found neccessary to move away from hardcoding strings directly into JDBC structures
as there were many optional components to some queries. This was resolved by use of jOOQ. A library which supports generating SQL(amongst several other features). I have used the sql generation features of jOOQ heavily in the servlets comprising the server side implementation, as the abstract syntax tree structure allows for easy addition and removal of filtering conditions without having to deal directly with a combinatorial explosion as more optional filters are added. This allows significant flexibility in filtering options, and easy extensibility. jOOQ also offers significant cross DBMS support, allowing the servlet code to work from a different database system with minimal effort.


\subsection{Security Concerns}
The server is constrained to use a very old version of java, as the system was developed against an older version of tomcat. Exposing this version of the server to the internet is not advised, as it may contain unpached vulnerabilities. However, porting to a later version of tomcat should not involve any great difficulty.  Configuration files would need to be rebuilt, however there should not be any significant compatibility issues.

\section{Testing}
Automated testing of user interfaces remains problematic, with many tools suffering from severe fragility where UI layout is modified. In most testing libraries, mous interaction is recorded at test design, then played back artificially on execution, This causes severe fragility as if a control or button moves the recorded mouse movements will miss the button, causing test failures. Further research is needed to find a testing library suitable for testing, however the test automation tool created by Dojo appears useful \cite{dojo2013test}. - automated testing not used. heavy use of in browser dev tools and debugger. manual testing. 