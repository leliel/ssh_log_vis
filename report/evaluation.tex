\chapter{Evaluation}\label{eval}

Evaluation of the tool produced in this project served to determine if there is sufficient merit to the idea to justify further development work on the tool. If the evaluation determines that there is sufficient potential to justify the development work, future work can evaluate the effectiveness of the design in a more rigorous manner. I have chosen to perform an exploratory evaluation due to time limitations on this project \cite{Ellis:2006:EAU:1168149.1168152}. It is not possible to achieve a representative sample of sufficient size to have predictive power in the time available. This limitation strongly suggests against attempting to demonstrate conclusively the usefulness and usability of the system. 

Ethics approval has been granted for this experimental design.

\section{Evaluation Design}

Evaluation was performed by means of a qualitative user study, with 6 participants. 
Participants were given up to ten minutes to familiarise themselves with the tool, and how it behaves. After familiarisation, participants were asked to answer four questions about each of two datasets. For each of the 8 combinations a brief questionnaire was completed indicating opinions about the tool's performance in the task\cite{lewis1995ibm}.
Questions were presented to users in randomised order. This was done in order to avoid learning effects distorting results for the first question.
\begin{enumerate}
\item{Find an instance of a successful brute force attack on root.}
\item{Find an instance of a successful scattergun attack.
(an instance where the attacker attempts many common username/password pairs at random).}
\item{Find an instance of a legitimate user logging in from an abnormal location.}
\item{Find an instance of a legitimate user logging in at an abnormal time.}
\end{enumerate}

Timing and accuracy for each task was recorded. 
Time was measured manually, by means of a stopwatch. While stopwatches are hard to use for subsecond accuracy, this is not required for a study of this type. Errors in timing on the order of 5 seconds are acceptable.

Tasks were graded for accuracy by 

\subsection{Users}

As this tool targets domain experts in the security domain, I will be recruiting participants from the security industry, both on campus and off. I have pitched the experiment to students enrolled in NWEN405, which deals with computer security as a secondary source of participants. Students are less ideal, as their experience and domain knowledge are more limited than those that have been working in the industry for some time.
--Expand this section, leave full personae in appendix.

\subsection{Datasets}

Two datasets have been acquired for use in this experiment.
\begin{enumerate}
\item{Honeynet Forensic challenge 10 dataset \cite{forensic10}. This dataset is anonymised and public domain. The data presented here covers a single server, with low traffic (9MB covering mar 16 to may 2)}
\item{Anonymised logs from the Engineering and Computer Science network at Victoria University. This dataset is significantly larger, covering 3 servers and 2 weeks of time on a high traffic network. Exact data size is not yet known, as the data collection process is ongoing at time of writing.}
\end{enumerate}

\subsection{Setup}
Participants were provided with a Dell Optiplex 9010 with an i7-3770 CPU and 8Gb of ram running Arch linux 3.7.5.
Browser used was google chrome 26.0.1410.63

The experiment was conducted in a quiet lab, with only the experimenter and subject present. Each question was allocated an 8 minute maximum, based on test runs with a supervisor.

\section{Results}

insert table of results here.. then do analysis after raw results shown and discussed.
note why time for participant 4 was discarded on one question (experimenter error.)
initial results suggest several improvements to be made
\begin{enumerate}
\item{Improved navigation through timeline. Users complain that it is easy to get lost when zooming. Animations may ease this transition. (possibly animate zooming in by expanding target bin to fill timeline, then chunking into new bins?)}
\item{Improved filtering of results by IP. several extra features are requested here, with the ability to filter by subnet, as well as hiding IP's (inverse of IP of Interest filter.)}
\item{Ability to filter connections by authtype, ie : excluding hostbased when looking for a successful brute force attack.}
\item{Improve performance of abnormal time and location detection, reduce false positives and spurious results by suppressing alerting on invalid or failed attempts. This will significantly cut down the number of abnormalities reported, reducing the false positive rate significantly.}
\end{enumerate}

include comparison with goals here.. how well did we meet them?
\begin{enumerate}
\item{Effective use of information hiding to prevent information overload.}
\item{Prevent masking of important data in noise or hiding. failed, more work required on filtering and clustering}
\item{Provide strong filtering and highlighting options. weaker than required, more work needed}
\item{Show surrounding context for anomalous accesses.}
\item{GeoIP support to add context to login attempts. geoIP results used in clustering algorithm, but not directly displayed.}
\item{Allow the user control over which machine is monitored at any given time. -included as dropdown selector, may have some scale issues for networks with many machines to monitor. no ability to select multiple machines currently, though dropdown easily expanded to allow this.}
\item{Show network context for currently monitored machine. dropped for lack of time}
\item{Performance capable of handling millions of events without perceptible delays. - untested, but unlikely due in part to network latency and javascript performance issues. did not have sufficiently large dataset to test performance with more than $10^3$ magnitude.}
\item{Extensible log parsing: don't prevent extension to other similar log types.}
\end{enumerate}
